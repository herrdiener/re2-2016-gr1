\documentclass{beamer}
\usepackage[british]{babel}
%\usepackage[utf8x]{inputenx}
%\usepackage[T1]{fontenc}

\mode<presentation> {

%\usetheme{default}
%\usetheme{AnnArbor}
%\usetheme{Antibes}
%\usetheme{Bergen}
%\usetheme{Berkeley}
%\usetheme{Berlin}
%\usetheme{Boadilla}
%\usetheme{CambridgeUS}
%\usetheme{Copenhagen}
%\usetheme{Darmstadt}
%\usetheme{Dresden}
%\usetheme{Frankfurt}
%\usetheme{Goettingen}
%\usetheme{Hannover}
%\usetheme{Ilmenau}
%\usetheme{JuanLesPins}
%\usetheme{Luebeck}
\usetheme{Madrid}
%\usetheme{Malmoe}
%\usetheme{Marburg}
%\usetheme{Montpellier}
%\usetheme{PaloAlto}
%\usetheme{Pittsburgh}
%\usetheme{Rochester}
%\usetheme{Singapore}
%\usetheme{Szeged}
%\usetheme{Warsaw}

%\usecolortheme{albatross}
%\usecolortheme{beaver}
%\usecolortheme{beetle}
%\usecolortheme{crane}
%\usecolortheme{dolphin}
%\usecolortheme{dove}
%\usecolortheme{fly}
%\usecolortheme{lily}
%\usecolortheme{orchid}
%\usecolortheme{rose}
%\usecolortheme{seagull}
%\usecolortheme{seahorse}
%\usecolortheme{whale}
%\usecolortheme{wolverine}

%\setbeamertemplate{footline} % To remove the footer line in all slides uncomment this line
%\setbeamertemplate{footline}[page number] % To replace the footer line in all slides with a simple slide count uncomment this line

%\setbeamertemplate{navigation symbols}{} % To remove the navigation symbols from the bottom of all slides uncomment this line
}

\usepackage{graphicx} % Allows including images
\usepackage{booktabs} % Allows the use of \toprule, \midrule and \bottomrule in tables
\usepackage{mathtools}

%----------------------------------------------------------------------------------------
%	TITLE PAGE
%----------------------------------------------------------------------------------------

\title[Automated Traceability]{Automated Techniques\\for Finding and Maintaining Traces} % The short title appears at the bottom of every slide, the full title is only on the title page

\author{Group 1}
\date{28th November 2016}

\begin{document}

\begin{frame}
\titlepage
\end{frame}

\begin{frame}
\frametitle{Overview} % Table of contents slide, comment this block out to remove it
\tableofcontents % Throughout your presentation, if you choose to use \section{} and \subsection{} commands, these will automatically be printed on this slide as an overview of your presentation
\end{frame}

%----------------------------------------------------------------------------------------
%	PRESENTATION SLIDES
%----------------------------------------------------------------------------------------

%------------------------------------------------
\section{Preliminary Remarks}
\begin{frame}
\frametitle{The Need for Automatisation}
Traditional requirements tracing is
\begin{itemize}
\item slow
\item expensive
\item prone to non-completeness (missing links)
\end{itemize}

\vfill$\Rightarrow$ Indicators for automation potential

\end{frame}

%------------------------------------------------

\section{Methodology}
\begin{frame}
\frametitle{How to Find Links?}

\begin{itemize}
\item Relevancy likelihood estimation is a basic task in computational linguistics.
\item Basically a search engine where everything's a query and everything's a document.
\item One possible technique: term vectors.
\end{itemize}
\end{frame}

\begin{frame}
\frametitle{Term Vectors}
How relevant is a specific term to an artifact?

\begin{block}{Term Relevancy Vector structure}
\[
\vec{r}_\text{doc}=
\begin{pmatrix*}[l]
  r_\text{turnstile}\\
  r_\text{card}\\
  r_\text{the}\\
  \vdots\\
  r_\text{operator}
\end{pmatrix*}
=
\begin{pmatrix}
  0.22\\
  0.68\\
  0.02\\
  \vdots\\
  0.00\\
\end{pmatrix}
\]
\end{block}

\vfill
How relevant are two artifacts to each other?

\begin{block}{Cosine Similarity}
  \[ rel_{\vec r_1, \vec r_2} = \cos(\vec r_1 \angle \vec r_2)
  = \frac{\vec r_1 \cdot \vec r_2}{|\vec r_1| \cdot |\vec r_2| } \]
\end{block}

\end{frame}

\begin{frame}
\frametitle{What to do with this information?}
\begin{itemize}
\item We have an automatically generated relevancy likelihood extimation for each pair of trace artifacts.
\item Two options:
  \begin{enumerate}
  \item Register all traces with a similarity $>\theta$ and be done with it.
  \item Present higher-probability links to human analyst for validation.
  \end{enumerate}
\item Option 2. Definitely option 2.
\item Therefore recall must be optimised over precision.
\end{itemize}
\end{frame}

\begin{frame}
\frametitle{Learning}
\begin{itemize}
\item Automatic relevancy vectors are unreliable.
\item Let's steal some stuff from computational linguists again!
  \begin{itemize}
  \item Weight matrices are useful!
  \end{itemize}
\end{itemize}

\begin{block}{Adding a weight vector}
\[\vec{r}_\text{doc}\cdot \vec w=
\begin{pmatrix*}[l]
  r_\text{turnstile}\\
  r_\text{card}\\
  r_\text{the}\\
  \vdots\\
  r_\text{operator}
\end{pmatrix*}
\cdot
\begin{pmatrix*}[l]
  w_\text{turnstile}\\
  w_\text{card}\\
  w_\text{the}\\
  \vdots\\
  w_\text{operator}
\end{pmatrix*}
=
\begin{pmatrix}
  0.22\\
  0.68\\
  0.02\\
  \vdots\\
  0.00\\
\end{pmatrix}
\cdot
\begin{pmatrix}
  0.75\\
  2.00\\
  0.25\\
  \vdots\\
  1.00\\
\end{pmatrix}
\]
\end{block}

\begin{itemize}
\item Weights can be uniformly initalised and later modified according to feedback from human analysts.
\end{itemize}

\end{frame}

%------------------------------------------------

\section{Studies}
\begin{frame}
\frametitle{Hayes et al. 2007}
\begin{itemize}
\item RETRO system
  \begin{itemize}
  \item Facilitates tracing for human analysts
  \item Intended for vertical tracing (`high-level' to `low-level' artifacts)
  \end{itemize}
\end{itemize}

\begin{table}
\begin{tabular}{l||l|l|r|}
  &Recall&Precision&Time [min]\\\hline\hline
  Manual group&0.33&0.24&120.67\\\hline
  RETRO group&0.70&0.13&41.88\\\hline
  T test (p value)&0.001&0.01&0.0004\\\hline
\end{tabular}
\end{table}

\end{frame}

%------------------------------------------------

\begin{frame}
\frametitle{Huang et al. 2007}
\begin{itemize}
\item Poirot System
  \begin{itemize}
  \item Smarter parsing
  \item (Programming) stopword removal
  \item Normalisation of {\tt conventionallyStyledNames();}
  \end{itemize}
\item Achieved 90\% recall and about 30\% precision
\item Deduced nine `best practices'
\end{itemize}
\end{frame}

\begin{frame}
\frametitle{Huang's Best Practices}
\begin{enumerate}
\item Trace for a purpose
\item Define a suitable trace granularity
\item Support in-place traceability
\item Use a well-defined project glossary
\item Write quality requirements
\item Construct a meaningful hierarchy
\item Bridge the intradomain semantic gap
\item Create rich content
\item \textbf{Use a process improvement plan}
\end{enumerate}
\end{frame}

%------------------------------------------------

\section{Discussion}
\begin{frame}
\Huge{\centerline{Discussion}}
\end{frame}

%----------------------------------------------------------------------------------------

\end{document} 

\documentclass{beamer}
\usepackage[british]{babel}
%\usepackage[utf8x]{inputenx}
%\usepackage[T1]{fontenc}

\mode<presentation> {

%\usetheme{default}
%\usetheme{AnnArbor}
%\usetheme{Antibes}
%\usetheme{Bergen}
%\usetheme{Berkeley}
%\usetheme{Berlin}
%\usetheme{Boadilla}
%\usetheme{CambridgeUS}
%\usetheme{Copenhagen}
%\usetheme{Darmstadt}
%\usetheme{Dresden}
%\usetheme{Frankfurt}
%\usetheme{Goettingen}
%\usetheme{Hannover}
%\usetheme{Ilmenau}
%\usetheme{JuanLesPins}
%\usetheme{Luebeck}
\usetheme{Madrid}
%\usetheme{Malmoe}
%\usetheme{Marburg}
%\usetheme{Montpellier}
%\usetheme{PaloAlto}
%\usetheme{Pittsburgh}
%\usetheme{Rochester}
%\usetheme{Singapore}
%\usetheme{Szeged}
%\usetheme{Warsaw}

%\usecolortheme{albatross}
%\usecolortheme{beaver}
%\usecolortheme{beetle}
%\usecolortheme{crane}
%\usecolortheme{dolphin}
%\usecolortheme{dove}
%\usecolortheme{fly}
%\usecolortheme{lily}
%\usecolortheme{orchid}
%\usecolortheme{rose}
%\usecolortheme{seagull}
%\usecolortheme{seahorse}
%\usecolortheme{whale}
%\usecolortheme{wolverine}

%\setbeamertemplate{footline} % To remove the footer line in all slides uncomment this line
%\setbeamertemplate{footline}[page number] % To replace the footer line in all slides with a simple slide count uncomment this line

%\setbeamertemplate{navigation symbols}{} % To remove the navigation symbols from the bottom of all slides uncomment this line
}

\usepackage{graphicx} % Allows including images
\usepackage{booktabs} % Allows the use of \toprule, \midrule and \bottomrule in tables

%----------------------------------------------------------------------------------------
%	TITLE PAGE
%----------------------------------------------------------------------------------------

\title[Distributed RE II]{Distributed Requirements Engineering II} % The short title appears at the bottom of every slide, the full title is only on the title page

\author{Group 1}
\date{31st October 2016}

\begin{document}

\begin{frame}
\titlepage
\end{frame}

\begin{frame}
\frametitle{Overview} % Table of contents slide, comment this block out to remove it
\tableofcontents % Throughout your presentation, if you choose to use \section{} and \subsection{} commands, these will automatically be printed on this slide as an overview of your presentation
\end{frame}

%----------------------------------------------------------------------------------------
%	PRESENTATION SLIDES
%----------------------------------------------------------------------------------------

%------------------------------------------------
\section{General remarks}
\begin{frame}
\frametitle{General remarks}
Sed iaculis dapibus gravida. Morbi sed tortor erat, nec interdum arcu. Sed id lorem lectus. Quisque viverra augue id sem ornare non aliquam nibh tristique. Aenean in ligula nisl. Nulla sed tellus ipsum. Donec vestibulum ligula non lorem vulputate fermentum accumsan neque mollis.\\~\\

Sed diam enim, sagittis nec condimentum sit amet, ullamcorper sit amet libero. Aliquam vel dui orci, a porta odio. Nullam id suscipit ipsum. Aenean lobortis commodo sem, ut commodo leo gravida vitae. Pellentesque vehicula ante iaculis arcu pretium rutrum eget sit amet purus. Integer ornare nulla quis neque ultrices lobortis. Vestibulum ultrices tincidunt libero, quis commodo erat ullamcorper id.
\end{frame}

%------------------------------------------------

\section{Papers}

\begin{frame}
\frametitle{Lloyd 2002: Effectiveness of Elicitation Techniques}
An experiment to simulate a distributed elicitation process.

\begin{itemize}
\item 6 `customer' and `engineer' teams
\item Free choice of elicitation techniques
\item Surveilled communication:
  \begin{itemize}
  \item 4$\times$90 minutes of audio conference
  \item A document sharing platform
  \item Email
  \end{itemize}
\item SRS graded by:
  \begin{itemize}
  \item Manual score
  \item Requirements evolution
  \item Requirements errors
  \item Original requirements included
  \end{itemize}
\end{itemize}
\end{frame}

\begin{frame}
\frametitle{Lloyd 2002: Findings}
\begin{itemize}
\item SRS quality coincided with perceived customer participation
\item SRS quality coincided with prior RE experience
\item Usage of questionnaires had a negative impact
\item Intensive use of the synchronous meetings yielded better quality
\end{itemize}
\end{frame}

%------------------------------------------------

\begin{frame}
\frametitle{Yang 2003: Web-based information systems}
Explores Web-based information systems (i.e.\ anything with a web component, which was a specific criterion at the time of publication), proposes a three-stage requirement analysis procedure:
\begin{enumerate}
\item Initial analysis (Regular purpose analysis)
\item Key user requirements elicitation (Identifying and questioning particularly relevant users to elicit additional requirements)
\item Regular user responses (validating step 2)
\end{enumerate}
\end{frame}

\begin{frame}
\frametitle{Yang 2003: Finding key users}
\begin{itemize}
\item Graph analysis (degree, betweenness, closeness)
\item Semantic analysis (gate keepers, opinion leaders, boundary spanners)
\end{itemize}
\end{frame}

%------------------------------------------------

\section{Discussion}
\begin{frame}
\Huge{\centerline{Discussion}}
\end{frame}

%----------------------------------------------------------------------------------------

\end{document} 
